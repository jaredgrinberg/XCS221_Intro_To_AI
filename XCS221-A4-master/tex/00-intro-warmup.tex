{\bf Warmup}

First, play a game of classic Pac-Man to get a feel for the assignment:

\begin{lstlisting}
(XCS221)$ python pacman.py
\end{lstlisting}

You can always add |--frameTime 1| to the command line to run in "demo mode"
where the game pauses after every frame.

Now, run the provided |ReflexAgent| in |submission.py|:

\begin{lstlisting}
(XCS221)$ python pacman.py -p ReflexAgent
\end{lstlisting}

Note that it plays quite poorly even on simple layouts:

\begin{lstlisting}
(XCS221)$ python pacman.py -p ReflexAgent -l testClassic
\end{lstlisting}

You can also try out the reflex agent on the default |mediumClassic| layout with
one ghost or two.

\begin{lstlisting}
(XCS221)$ python pacman.py -p ReflexAgent -k 1
(XCS221)$ python pacman.py -p ReflexAgent -k 2
\end{lstlisting}

{\em Note: You can never have more ghosts than the layout permits (see
|src/layouts/mediumClassic.lay|).}

{\em Options: Default ghosts are random; you can also play for fun with slightly
smarter directional ghosts using |-g DirectionalGhost|. You can also play
multiple games in a row with |-n|. Turn off graphics with |-q| to run lots of
games quickly.}

Now that you are familiar enough with the interface, inspect the |ReflexAgent|
code carefully (in |submission.py|) and make sure you understand what it's
doing. The reflex agent code provides some helpful examples of methods that
query the |GameState|: A |GameState| object specifies the full game state,
including the food, capsules, agent configurations, and score changes: see
|submission.py| for further information and helper methods, which you will be
using in the actual coding part. We are giving an exhaustive and very detailed
description below, for the sake of completeness and to save you from digging
deeper into the starter code. The actual coding part is very small -- so please
be patient if you think there is too much writing.

{\em Note: If you wish to run the game in the terminal using a text-based
interface, check out the |terminal| directory.}
