\item \points{2b}
Now let's do the general case in code: implement |get_sum_variable()|, which
takes in a sequence of non-negative integer-valued variables and returns a
variable whose value is constrained to equal the sum of the variables. You will
need to access the domains of the variables passed in, which you can assume
contain only non-negative integers. The parameter |maxSum| is the maximum sum
possible of all the variables. You can use this information to decide the proper
domains for your auxiliary variables.

How can this function be useful? Suppose we wanted to enforce the constraint
$[X_1 + X_2 + X_3 \le K]$. We would call |get_sum_variable()| on $(X_1,X_2,X_3)$
to get some auxiliary variable $Y$, and then add the constraint $[Y \le K]$.
Note: You don't have to implement the $\le$ constraint for this part.
