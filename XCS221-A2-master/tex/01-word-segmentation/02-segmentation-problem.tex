\item \points{1b}
Implement an algorithm that, unlike the greedy algorithm, finds the optimal word
segmentation of an input character sequence. Your algorithm will consider costs
based simply on a unigram cost function. |UniformCostSearch| (UCS) is
implemented for you in |util.py|, and you should make use of it here.
\footnote{Solutions that use UCS ought to exhibit fairly fast execution time for
this problem, so using A* here is unnecessary.}

Before jumping into code, you should think about how to frame this problem as a
state-space {\bf search problem}.  How would you represent a state?  What are
the successors of a state?  What are the state transition costs?  (You don't
need to answer these questions in your writeup.)

Fill in the member functions of the |SegmentationProblem| class and the
|segmentWords| function.

The argument |unigramCost| is a function that takes in a single string
representing a word and outputs its unigram cost. You can assume that all of the
inputs would be in lower case.

The function |segmentWords| should return the segmented sentence with spaces
as delimiters, i.e. |' '.join(words)|.

To request a segmentation, type |seg mystring| into the prompt.  For example:

\begin{lstlisting}
$ python submission.py
Training language cost functions [corpus: leo-will.txt]... Done!

>> seg thisisnotmybeautifulhouse

Query (seg): thisisnotmybeautifulhouse

this is not my beautiful house
\end{lstlisting}

{\em Hint: You are encouraged to refer to |NumberLineSearchProblem| and
|GridSearchProblem| implemented in |util.py| for reference. They don't
contribute to testing your submitted code but only serve as a guideline for what
your code should look like.}

{\em Hint: The actions that are valid for the |ucs| object can be
accessed through |ucs.actions|.}
